\documentclass[10pt,a4paper,oneside]{article}
\usepackage[utf8]{inputenc}
\usepackage[english,russian]{babel}
\usepackage{amsmath}
\usepackage{amsthm}
\usepackage{amssymb}
\usepackage{cmll}
\usepackage{enumerate}
\usepackage{stmaryrd}
\usepackage[left=2cm,right=2cm,top=2cm,bottom=2cm,bindingoffset=0cm]{geometry}
\usepackage{proof}
\usepackage{url}
\usepackage{CJKutf8}
\newcommand{\gq}[1]{\texttt{<<}#1\texttt{>>}}
\newcommand{\ogq}[1]{\overline{\texttt{<<}#1\texttt{>>}}}
\begin{document}

\begin{center}{\Large\textsc{\textbf{Теоретические (``малые'') домашние задания}}}\\
             \it Теория типов, ИТМО, М3334-М3339, осень 2018 года\end{center}

\section*{Домашнее задание №1: <<знакомство с лямбда-исчислением>>}

\begin{enumerate}
\item Расставьте скобки:

$\lambda f.\lambda x.f\ x\ \ (\lambda c.g\ f)\ x\ a\ \lambda b.\lambda a.x$

\item Проведите бета-редукции и приведите выражения к нормальной форме:

\begin{enumerate}
\item $(\lambda f.\lambda x.f\ (f\ x))\ (\lambda f.\lambda x.f\ (f\ x))$
\item $(\lambda a.\lambda b.b)\ ((\lambda x.x\ x)\ (\lambda x.x\ x\ x))$
\end{enumerate}

\item Выразите следующие функции в лямбда-исчислении:

\begin{enumerate}
\item Or, Xor
\item isZero (T, если аргумент равен 0, иначе F)
\item isEven (T, если аргумент чётный)
\item умножение на 2, умножение
\item возведение в степень
\item вычитание 1, вычитание
\end{enumerate}
\end{enumerate}

\section*{Домашнее задание №2: <<пропущенные теоремы лямбда-исчисления>>}

Докажите следующие леммы, упомянутые, но недоказанные на лекции:

\begin{enumerate}
\item Если отношение $R$ обладает ромбовидным свойством, то и отношение $R^*$ 
(транзитивное и рефлексивное замыкание $R$) также им обладает.
\item Отношение альфа-эквивалентности является отношением эквивалентности.
\item Если $P_1\rightrightarrows_\beta P_2$ и $Q_1\rightrightarrows_\beta Q_2$, то $P_1[x := Q_1] \rightrightarrows_\beta P_2[x := Q_2]$.
\item $(\rightrightarrows_\beta)$ обладает ромбовидным свойством.
\item $(\rightrightarrows_\beta) \subseteq (\rightarrow_\beta)^*$ 
\item $(\rightarrow_\beta) \subseteq (\rightrightarrows_\beta)$ 
\end{enumerate}

\section*{Домашнее задание №3: <<просто типизированное лямбда-исчисление>>}

\begin{enumerate}
\item Докажите лемму о промежуточных типах (Generation lemma, 3.1.6 из 
Morten Heine B. Sørensen, Pawel Urzyczyn: Lections on the Curry-Howard Isomorphism). 
А именно, покажите, что:
  \begin{enumerate}
  \item $\Gamma \vdash x : \tau$ влечёт $x : \tau \in \Gamma$.
  \item $\Gamma \vdash M N : \sigma$ влечёт существование типа $\tau$, такого, что
    что $\Gamma \vdash M : \tau\rightarrow\sigma$ и $\Gamma \vdash N : \tau$.
  \item $\Gamma \vdash \lambda x.M : \sigma$ влечёт существование типов $\tau$ и $\rho$, таких, что
    $\Gamma, x: \tau \vdash M : \rho$ и $\sigma = \tau\rightarrow\rho$
  \end{enumerate}

\item Докажите лемму о подстановке (Substitution lemma, 3.1.8):
\begin{enumerate}
   \item Обозначим за $\sigma[\alpha:=\tau]$ (за $\Gamma[\alpha:=\tau]$) замену всех элементарных 
   типов $\alpha$ на тип $\tau$ в типе $\sigma$ (во всех типах в $\Gamma$). 
   Тогда, если $\Gamma \vdash M : \sigma$, то $\Gamma[\alpha := \tau] \vdash M : \sigma [\alpha := \tau]$.
   \item Если $\Gamma, x:\tau \vdash M : \sigma$ и $\Gamma \vdash N : \tau$, то $\Gamma \vdash M[x := N]:\sigma$.
\end{enumerate}

\item Докажите лемму о редукции терма (Subject reduction proposition, 3.1.9):
    если $\Gamma\vdash M:\sigma$ и $M \rightarrow_\beta N$, то $\Gamma \vdash N:\sigma$.

\item Пользуясь предыдущими пунктами, покажите, что $Y$ нетипизируем в просто типизированном лямбда-исчислении.

\item Найдите терм $M$ и два различных типа $\sigma$ и $\tau$, что $\vdash M: \sigma$ и $\vdash M:\tau$.
    А существует ли терм $M$, имеющий в точности один тип?

\item Покажите, что лемма о редукции терма не работает <<в обратную сторону>>. А именно, что:
   \begin{enumerate}
   \item Найдутся $M$, $N$ и $\tau$, что $\vdash N:\tau$, $M \rightarrow_\beta N$, но $M$ не имеет типа.
   \item Найдутся $M$, $N$, $\sigma$ и $\tau$, что $\vdash M:\sigma$, $\vdash N:\tau$ и $M \rightarrow_\beta N$, но
     $\nvdash M:\tau$.
   \end{enumerate}
\end{enumerate}

\section*{Домашнее задание №4: <<просто типизированное лямбда-исчисление; алгебраические типы>>}

\begin{enumerate}
\item Списки и алгебраические типы.
  В данном задании потребуется строить и преобразовывать довольно сложные лямбда-выражения.
  Для проверки рекомендуем пользоваться интерпретатором, например, можно взять \verb!LCI!:
  \url{https://chatziko.github.io/lci/}. Возможно, для демонстрации домашнего задания
  вам потребуется использовать свой ноутбук и проектор.

  \begin{enumerate}
  \item Определите алгебраический тип для списка целых чисел в вашем любимом языке программирования.
        На Окамле это будет \verb!type int_list = Nil | Cons of (int * int_list)!. Вы можете использовать
        и не функциональный язык (C++, Kotlin и т.п.), но вы должны применять именно алгебраический тип
        или его аналог (то есть, \verb!union! в C++, \verb!sealed class! в Kotlin и т.п.).
  \item Напишите функции вычисления длины списка, подсчёта суммы списка, произведения списка.
  \item Определите функцию высшего порядка \verb!map! (применяющую переданную параметром функцию 
        к каждому элементу списка), и примените её для построения списка нулей 
        (превратить список чисел в список нулей той же длины), удвоенных значений
        (превратить список $[1,3,5]$ в $[2,6,10]$), списка остатков от деления на 2 ($[2,3,5]$ в $[0,1,1]$).
  \item Перепишите весь код из предыдущих пунктов в чистых лямбда-выражениях, используя рассмотренные
        на лекции представления в лямбда-исчислении для упорядоченных пар и алгебраических типов.
  \end{enumerate}

\item Ещё немного алгебраических типов.
  Аналогично предыдущему пункту, определите на языке высокого уровня алгебраический тип для корней 
  квадратного уравнения. Варианты значений: <<нет решений>> без параметров, <<одно решение>> 
  с одним параметром, <<два решения>> с двумя параметрами.
  Определите функции вычисления корней по коэффициентам квадратного уравнения и печати корней.

\item Деревья с помощью алгебраических типов.
  Определите на языке высокого уровня тип для дерева двоичного поиска, 
  варианты для узла: <<лист>> без параметров и <<ветвь>> с двумя сыновьями и целочисленным значением.
  Определите:
  \begin{enumerate}
    \item функцию печати дерева;
    \item функцию поиска значения в дереве;
    \item функцию добавления значения в дерево двоичного поиска;
    \item функцию удаления значения из дерева.
  \end{enumerate}

\item Доопределите бета-редукцию для просто типизированного лямбда-исчисления по Чёрчу.

\item Докажите теорему Чёрча-Россера для просто типизированного лямбда-исчисления по Чёрчу.

\item Докажите лемму о поднятии с лекции, а именно, что для всех $M,N \in \Lambda_{\texttt{к}}$:
  \begin{enumerate}
  \item если $M \rightarrow_\beta N$, то для любого $M' \in \Lambda_{\texttt{ч}}$, такого, что
        $|M'| = M$, найдётся такой $N' \in \Lambda_{\texttt{ч}}$, что $|N'| = N$ и $M' \rightarrow_\beta N'$;
  \item если $\Gamma \vdash M : \sigma$, то найдётся такой $M' \in \Lambda_{\texttt{ч}}$, что $\Gamma \vdash_\texttt{ч} M': \sigma$.
  \end{enumerate} 

\end{enumerate}

\section*{Домашнее задание №5: <<выразительная сила просто типизированного лямбда-исчисления, алгоритм унификации>>}

\begin{enumerate}
\item Покажите, что чёрчевский нумерал в общем случае имеет 
тип $(\alpha\rightarrow\alpha)\rightarrow(\alpha\rightarrow\alpha)$.
Имеют ли нумералы для $0$, $1$ и $2$ какие-то более общие типы?

\item Обозначим тип для целых чисел $\eta = (\alpha\rightarrow\alpha)\rightarrow(\alpha\rightarrow\alpha)$.
В данных обозначениях покажите, что операция сложения имеет тип $\eta\rightarrow\eta\rightarrow\eta$.

\item Напомним, что $\overline{m} = \lambda f.\lambda x.f^{(m)}\ x$ (чёрчевский нумерал для $m$).
Рассмотрим выражение $Power = \lambda m.\lambda n.n\ m$.
\begin{enumerate}
\item найдите тип $Power$;
\item покажите, что $Power\ \overline{m}\ \overline{n} = \overline{m^n}$, найдите тип $Power\ \overline{m}\ \overline{n}$;
\item покажите, что $\lambda x.Power\ x\ x$ не имеет типа;
\item объясните кажущееся противоречие между предыдущими пунктами: почему 
$Power\ \overline{2}\ \overline{2}$ имеет тип, а $(\lambda x.Power\ x\ x)\ \overline{2}$ не имеет типа.
\end{enumerate}

\item Докажите, что изложенный на лекции алгоритм унификации всегда завершается.
Указание: постройте оценку сложности уравнения в алгебраических термах и покажите,
что эта оценка уменьшается при каждом шаге алгоритма.

\end{enumerate}

\section*{Домашнее задание №6: <<унификация и типы, комбинаторы>>}

\begin{enumerate}
\item Выразите $\lambda f.\lambda x.f\ x$ через $S$ и $K$.
\item Докажите, что алгоритм устранения абстракций $T$ с лекции, преобразующий замкнутое лямбда-выражение в 
выражение в комбинаторах $S$ и $K$, корректен. То есть, для любого лямбда-выражения $A$: 
\begin{enumerate}
\item $T(A)$ определено и вычисляется за конечное время;
\item $T(A) =_\beta A$;
\item если $A$ замкнуто, то $T(A)$ не содержит абстракций и свободных переменных и состоит только из
применений (аппликаций) и комбинаторов $S$ и $K$.
\end{enumerate}
\item Покажите, что базис $B$, $C$, $K$, $W$ позволяет выразить любое замкнутое лямбда-выражение.
\item Постройте систему аксиом для импликационного фрагмента просто типизированного лямбда-исчисления на основе
базиса $B$, $C$, $K$, $W$.
\item Будем говорить, что тип $\sigma$ есть частный случай типа $\theta$ (и записывать это как
$\sigma \sqsubseteq \theta$), если существует такая подстановка $S$, что $\sigma = S(\theta)$.
Рассмотрим лямбда-выражение $M$, такое, что $\vdash M: \sigma$ и $\vdash M: \theta$.
\begin{enumerate}
\item Покажите, что найдётся тип $\tau$, что $\vdash M:\tau$, $\sigma\subseteq\tau$ и $\theta\subseteq\tau$.
\item Всегда ли найдётся $\tau$, что $\tau \subseteq \sigma$ и $\tau \subseteq \theta$?
\item Всегда ли выполнено либо $\theta \subseteq \sigma$, либо $\sigma \subseteq \theta$?
\item Можно ли определить решётку на типах для данного $M$ с определённым выше отношением предпорядка $(\subseteq)$?
Естественно, вам потребуется рассмотреть классы эквивалентности типов, чтобы <<склеить>> случаи 
типов, отличающихся только переименованием переменных.
Какими свойствами эта решётка будет обладать (дистрибутивность, импликативность, существование 0 и т.п.)?
\end{enumerate}
\end{enumerate}

\section*{Домашнее задание №7: <<исчисление второго порядка, система F>>}

\begin{enumerate}
\item Докажите, что введённые на лекции представления для связок соответствуют правилам для связок:

\begin{enumerate} 
\item Конъюнкция. 
Если $\varphi \& \psi \equiv \forall \alpha.(\varphi\rightarrow\psi\rightarrow\alpha)\rightarrow\alpha$,
то всегда можно доказать заключения следующих правил при доказанных посылках:

\begin{center}\begin{tabular}{lp{0.5cm}lp{0.5cm}l}
$$\infer{\Gamma\vdash\varphi\&\psi}{\Gamma\vdash\varphi\quad\Gamma\vdash\psi}$$ &&
$$\infer{\Gamma\vdash\varphi}{\Gamma\vdash\varphi\&\psi}$$ &&
$$\infer{\Gamma\vdash\psi}{\Gamma\vdash\varphi\&\psi}$$
\end{tabular}\end{center}

\item Дизъюнкция.
Если $\varphi \vee \psi \equiv \forall \alpha.(\varphi\rightarrow\alpha)\rightarrow(\psi\rightarrow\alpha)\rightarrow\alpha$,
то можно показать такие правила:
\begin{center}\begin{tabular}{lp{0.5cm}lp{0.5cm}l}
$$\infer{\Gamma\vdash\varphi\vee\psi}{\Gamma\vdash\varphi}$$ &&
$$\infer{\Gamma\vdash\varphi\vee\psi}{\Gamma\vdash\psi}$$ &&
$$\infer{\Gamma\vdash\pi}{\Gamma,\varphi\vdash\pi\quad\Gamma,\psi\vdash\pi\quad\Gamma\vdash\varphi\vee\psi}$$
\end{tabular}\end{center}

\item Квантор существования.
Если $\exists \alpha.\varphi \equiv \forall \theta.(\forall\alpha.\varphi\rightarrow\theta)\rightarrow\theta)$, то
можно показать и следующие правила:

\begin{center}\begin{tabular}{lp{1cm}l}
$$\infer{\Gamma\vdash\exists \alpha.\varphi}{\Gamma\vdash\varphi[\alpha := \theta]}$$ &&
$$\infer[(\alpha\notin FV(\Gamma,\psi))]{\Gamma\vdash\psi}{\Gamma\vdash\exists\alpha.\varphi\quad\Gamma,\varphi\vdash\psi}$$
\end{tabular}\end{center}

\end{enumerate}
\end{enumerate}

\section*{Домашнее задание №8: <<экзистенциальные типы, ранг типа>>}

\begin{CJK}{UTF8}{min}
\begin{enumerate}

\item Докажите, что если $\sigma \in R(1)$, то существует $\psi$ с поверхностными кванторами, что
$\psi \vdash \sigma$ и $\sigma \vdash \psi$.

\item Докажите, что если $M:\sigma$ и $\sigma \in R(1)$, то существует $\psi$ с поверхностными
кванторами и лямбда-выражения $N$, $F$ и $B$ в системе F, такие, что $N:\psi$, 
$F:\sigma\rightarrow\psi$ и $B:\psi\rightarrow\sigma$.

\item Пусть задан экзистенциальный тип $\exists\alpha.\sigma$. Если рассматривать экзистенциальные
типы в терминах АТД, то $\alpha$ здесь --- переменная для типа структуры данных (массив, список и т.п.),
а $\sigma$ --- интерфейс (например, рассматривавшийся на лекции интерфейс стэка: 
$\alpha \with (\alpha\rightarrow\nu\rightarrow\alpha) \with (\alpha\rightarrow(\nu\with\alpha))$).

Также пусть задана реализация этого типа (актуальный тип для структуры данных --- $\beta$, и
актуальная реализация интерфейса --- $T$).

\begin{enumerate}
\item 
Операция \emph{упаковки}: обозначим за $$\texttt{pack}\ \{\beta,T\}\ \texttt{to}\ \exists\alpha.\sigma$$ выражение 
$$\Lambda \gamma.\lambda x: (\forall \alpha. \sigma\rightarrow\gamma).x\ \beta\ T$$

Покажите выполнение правила
$$\infer{\Gamma\vdash\texttt{pack}\ \{\beta,T\}\ \texttt{to}\ \exists\alpha.\sigma : \exists\alpha.\sigma}
{\Gamma\vdash T : \sigma[\alpha := \beta]}$$

\item Операция \emph{распаковки}: обозначим за $$\texttt{unpack}\ \{\alpha,x:\sigma\} = T\ \texttt{in}\ E^\delta$$ выражение
$$T\ \delta\ (\Lambda \alpha.\lambda x: \sigma.E)$$

Покажите выполнение правила
$$\infer{\Gamma\vdash \texttt{unpack}\ \{\alpha,x:\sigma\} = T\ \texttt{in}\ E^\delta: \delta}
{\Gamma\vdash T : \exists\alpha.\sigma \quad \Gamma,x:\sigma \vdash E : \delta}$$

\item Пункт убран.
%Выразите алгебраический тип от двух аргументов через экзистенциальный тип:
%постройте лямбда-выражения для $\texttt{in}_L$, $\texttt{in}_R$ и 
%\texttt{case}, использующие \texttt{pack} и \texttt{unpack},
%покажите, что правила вывода для дизъюнкции будут выполняться
%и что результат бета-редукции выражений в правилах будет ожидаемым.

\end{enumerate}

\item Рассмотрим абстрактный тип данных \verb!NatNum! (<<натуральное число>>), с операциями: (a) константа 0;
(б) сравнение двух значений; (в) прибавление 1; (г) печать в строку. Задайте его на языке Хаскель
с помощью экзистенциальных типов, предусмотрите три реализации (через \verb!Integer!, аксиоматику
Пеано и в виде строчек из цифр), напишите функции: (一) печати списка чисел в строку
(\verb![NatNum] -> String!); (二) удвоения каждого числа в списке (\verb![NatNum] -> [NatNum]!);
(三) суммы списка (\verb![NatNum] -> NatNum!).

\emph{Указание:} напомним, что значение экзистенциального типа --- это функция, берущая на вход
функцию, проводящую вычисление с абстрактным типом данных. Этот абстрактный тип доступен только
внутри вызова, наружу его передать нельзя. 
Поэтому, чтобы увидеть результат исполнения функций (一), (二) и (三), нужно также обернуть 
их в какое-то вычисление, выдающее 
<<не-абстрактный>> результат (например, текстовую строку), и уже это вычисление передавать
внутрь значения экзистенциального типа.

Смотрите пример реализации \url{https://github.com/shd/tt2018/blob/master/existential.hs}

\end{enumerate}
\end{CJK}

\section*{Домашнее задание №9: <<Idris, введение>>}
\begin{enumerate}
\item Определите в языке Идрис конъюнкцию и дизъюнкцию.
Определите все стандартные операции для них (инъекции, проекции и т.п.):
эти операции, очевидно, будут доказательством некоторых утверждений в интуиционистской
логике. Какие это утверждения, приведите их.

\item Определите функцию \verb!swap: Vect n a -> (Fin n) -> (Fin n) -> Vect n a!,
строящую новый вектор, в котором два элемента вектора поменяны местами.

\item Определите функции арифметики для Fin:\\
(а) \verb!plus_fin: Fin a -> Fib b -> Fin (a+b)!\\
(б) \verb!mul_fin: Fin a -> Fin b -> Fin (a*b)!\\
(в) \verb!dec_fin: Fin (S a) -> Fin a!

\item Определите функции минимума для натуральных (Пеано) и конечных чисел:\\
\verb!nat_min: Nat -> Nat -> Nat!\\
\verb!min_fin: {a:Nat} -> {b:Nat} -> Fin (S a) -> Fin (S b) -> Fin (S (nat_min a b))!

Также определите функции:
\begin{verbatim}
map2: {X:Type} -> {Y:Type} -> {Z:Type} -> (a:Nat) -> (b:Nat) -> 
                       (X->Y->Z) -> Vect a X -> Vect b Y -> Vect (nat_min a b) Z

index2: {X:Type} -> {Y:Type} -> (a:Nat) -> (b:Nat) -> Fin (nat_min a b) ->
                       Vect a X -> Vect b Y -> (X,Y)
\end{verbatim}

\end{enumerate}

\section*{Домашнее задание №10: <<Идрис, простые доказательства>>}
Ещё раз напомним основную идею доказательства:
доказать утверждение $\sigma$ --- это построить такое выражение $M$,
что $\vdash M:\sigma$.

В языках с типовой системой Хиндли-Милнера существует алгоритм, разрешающий
задачу типизации: там для вывода типов достаточно самого выражения $M$ и 
типа $\sigma$. В Идрисе, с учётом сложности языка, задача типизации неразрешима,
поэтому компилятор может догадаться до типа сам только в очень простых случаях.
В остальных ситуациях компилятору нужны подсказки.

Например, заметить, что \verb!Refl: 0+z=z+0!, компилятор сам не может
(сразу напомним, что если тип значений компилятору известен, то мы можем
писать \verb!+! вместо \verb!plus!, \verb!0! вместо \verb!Z!,
\verb!1! вместо \verb!S Z!).
Ему можно это пояснить, написав функцию, проводящую доказательство
по индукции, на лекции мы рассматривали такой её вариант:

\begin{verbatim}
plus_zero_commutative : (a : Nat) -> (0+a) = (a+0)
plus_zero_commutative Z = Refl
plus_zero_commutative (S k) = rewrite (plus_zero_commutative k) in Refl
\end{verbatim}

Однако, магия конструкции \verb!rewrite! оставила много вопросов,
поэтому давайте копнём глубже, и разберём этот пример с точки зрения 
функции \verb!replace!, которую использует \verb!rewrite!.

\verb!replace: (x=y) -> P x -> P y!.

Функция replace берёт два явных параметра, и один неявный ($P$).
$P$ --- это некоторый тип, зависящий от $x$. Функция имеет естественный 
смысл: если два значения равны, и свойство выполнено для одного из них, 
то оно выполнено и для другого.

Неявность $P$ предполагает, что компилятор может догадаться до того, что
это за значение, но на практике он не справляется с этой задачей. 
Поэтому обычно $P$ нужно указывать.

Давайте поймём, что это должен быть за $P$. В случае \verb!plus_zero_commutative!
мы, имея предположением индукции \verb!0+a=a+0!, должны
доказать \verb!0 + (S a) = (S a) + 0!. Логично взять предположение
за равенство $x=y$, при этом \verb!x! будет соответствовать \verb!0 + a!,
а \verb!y! будет соответствовать \verb!a + 0!.
Теперь подберём такое $P$, чтобы \verb!P (0 + a)! 
унифицировалось с \verb!0 + (S a)!, а \verb!P (a + 0)! унифицировалось 
с \verb!(S a) + 0!.

Давайте возьмём \verb!P = \w => S (0+a)=S w!. Тогда \verb!P (0 + a)! --- это
\verb!S (0+a) = S (0+a)! (что очевидно доказывается при помощи \verb!Refl!), 
а \verb!P (a+0)! --- это \verb!S (0+a) = S (a+0)! (что является требуемым 
утверждением, так как \verb!S (0+a)! = \verb!S(a)! = \verb!0+S(a)!; компилятор,
как мы обсуждали, способен короткие цепочки подобных преобразований
производить самостоятельно).

Итак, мы получили следующий код:
\begin{verbatim}
plus_zero_commutative: (a:Nat) -> 0 + a = a + 0
plus_zero_commutative Z = Refl
plus_zero_commutative (S a) = 
      replace {P = \w => S (0+a)=S w} (plus_zero_commutative a) Refl
\end{verbatim}

В отличие от \verb!replace!, конструкция \verb!rewrite! имеет дополнительный
эвристический алгоритм для подбора соответствующего $P$, поэтому в части случаев 
мы можем довериться её интеллекту.

\begin{enumerate}
\item Свойства равенства. Докажите, что:
\begin{enumerate}
\item \verb!x = y -> y = x!
\item \verb!x = y -> y = z -> x = z!
\item (конгруэнтность) \verb!(P: A->B) -> x = y -> P x = P y!
\end{enumerate}

\item Простая арифметика --- сложение. Докажите, что:
\begin{enumerate} 
\item \verb!x = x + 0!
\item \verb!S x = 1 + x!
\item \verb!S x = x + 1!
\item \verb!S x + S x = S (S (x + x))!
\item \verb!S x + S y = S (S (x + y))!
\item \verb!S (x + y) = x + (S y)!
\item \verb!x + y = y + x!
\item \verb!x + (y + z) = (x + y) + z!
\end{enumerate}

\item Простая арифметика --- умножение. Докажите, что:
\begin{enumerate} 
\item \verb!0 = x * 0!
\item \verb!0 = 0 * x!
\item \verb!x = 1 * x!
\item \verb!x = x * 1!
\item \verb!x * y = y * x!
\item \verb!x * (y * z) = (x * y) * z!
\item \verb!x * (y + z) = x * y + x * z!
\end{enumerate}

\item Отношение <<меньше или равно>> определено в библиотеке Идрис так:

\begin{verbatim}
data LTE : (n : Nat) -> (m : Nat) -> Type where
    LTEZero : LTE 0 right  -- Zero is the smallest Nat
    LTESucc : LTE left right -> LTE (S left) (S right)
                           -- If n <= m, then n + 1 <= m + 1
\end{verbatim}

\begin{enumerate}
\item Докажите, что \verb!LTE 3 5!
\item Докажите, что \verb!LTE x y -> LTE x (S y)!
\item Докажите, что \verb!LTE x y -> LTE (x+n) (y+n)!
\end{enumerate}

\item Определите отношение <<строго больше>>, $GT$. Докажите, что

\verb!(x:Nat) -> (y:Nat) -> Either (LTE x y) (GT x y)!

\item Определите ограниченное вычитание \verb!sub! (\verb!sub x y! равно 0, если $x < y$), 
докажите:
\begin{enumerate}
\item \verb!LTE x y -> sub x y = 0!
\item \verb!sub x y = 0 -> LTE x y!
\item \verb!LTE y x -> y + (sub x y) = x!
\end{enumerate}

\end{enumerate}

\section*{Домашнее задание №11: <<Идрис, $\Sigma$ и $\Pi$-типы>>}

Данное домашнее задание предлагается для того, чтобы желающие могли самостоятельно
разобраться в заключительных темах курса, не разобранных на практике.

\begin{enumerate}
\item Рассмотрим обобщённую типовую систему $\lambda\omega$ (сильная омега).
\begin{enumerate}
\item Рассмотрим тип в Хаскеле: \verb!data Either a b = Left a | Right b!.
Предложите его реализацию в обобщённой типовой системе. 
Выразите конструкторы \verb!Left! и \verb!Right! в ней, укажите (и докажите) их тип.
\item Рассмотрим функции \verb!curry f = \x->\y->f (x,y)! и 
\verb!uncurry f = \(x,y) -> f x y!. Выразите их и их типы в обобщённой типовой системе, докажите их типы.
\item Выразите тип Y-комбинатора через $\Pi$-тип (будем считать, что лямбда-исчисление
расширено данным комбинатором).
\item Выразите функцию \verb!repeat n f x = if n > 0 then f (repeat (n-1) f x) else x!
и укажите её тип. Требуется ли Y-комбинатор для её задания?
\end{enumerate}

\item Теперь рассмотрим $\lambda C$ и язык Идрис.
\begin{enumerate}
\item Рассмотрим тип равенства. Выразите его в $\lambda C$, укажите и докажите его тип.
Напомню, что равенство --- это обобщённый алгебраический тип.
\item Выразите в $\lambda C$ функцию $(+)$ для целых чисел, укажите и докажите её тип.
\item Выразите в $\lambda C$ доказательства $a = 0+a$ и $a = a + 0$ (аналогично, выведите тип этих
доказательств).
\item Расмотрим тип \verb!Vect a x! из языка Идрис. Перенесите его определение в <<чистое>> лямбда-
исчисление в выбранной обобщённой типовой системе, укажите (и докажите) его тип.
\item Рассмотрим функцию \verb!head: (a: Num) -> (x: Type) -> Vect (S a) x -> x!, возвращающую
первый элемент вектора. Выразите её сигнатуру в $\lambda C$. Перенесите её определение в $\lambda C$,
докажите тип.
\item Укажите сигнатуру в $\lambda C$ для\\
\verb!filter: (a: Num) -> (x: Type) -> (x -> Bool) -> Vect a x -> (b: Num ** Vect b x)!.
Выразите её в данной типовой системе. Докажите её тип.
\item Докажите в Идрисе, что если $\exists y.x = S(S(y))$, то $\exists y.x = y + 2$.
Переведите это доказательство в $\lambda C$, выведите тип доказательства.
\end{enumerate}
\end{enumerate}

\section*{Задание на формальное доказательство при помощи Идрис}

\begin{enumerate}
\item Задание для сдающих зачёт до 2019.01.12 (8 баллов).

Формализуйте и докажите, что какими бы ни были два натуральных числа $a$ и $b$, если $a$ делится на $b$ и $b$ делится на $a$,
то $a=b$.

\item Задание для всех (20 баллов).

Формализуйте и докажите малую теорему Ферма: если $a$ ($a \in \mathbb{N}$) не делится на простое число $p$,
то $$a^{p-1} \equiv 1 \mod p$$

\item Задание для всех (20 баллов).

Формализуйте в языке Идрис и докажите следующее утверждение: связный неориентированный
граф имеет Эйлеров цикл тогда и только тогда, когда каждая вершина в нём имеет чётную степень.

\item Задание для сдававших зачёт 2019.01.21 (8 баллов).

Докажите, что $x+x$ делится на 2 --- то есть, существует такое $k$, что $2k=x+x$.

\item Задание для сдававших зачёт 2019.01.25 (8 баллов).

Докажите, что если $a$ делится на $b$ и $b$ делится на $c$, то $a$ делится на $c$.

\item Задание для сдающих зачёт 2019.01.28 (8 баллов).

Докажите, что если $1$ делится на натуральное $a$ ($\exists t. at = 1$), то $a=1$.

\item Задание для всех, желающих набрать баллы, сдавших <<свою>> задачу (8 баллов).

Пусть знак $(-)$ означает <<ограниченное вычитание>> (если $a<b$, то $a-b=0$).
Докажите, что $(a-b)^2 - (a+b)^2 = 0$.

\item Задание для сдававших зачёт 2019.02.01--2019.02.22 (8 баллов)

Докажите, что если $a > b$ ($a$ и $b$ --- натуральные), то $a^2 > b^2$.

\item Задание для всех, желающих набрать баллы, сдавших <<свою>> задачу (8 баллов).

Докажите, что $\max(a,\min(a,b)) = a$. 

\item Задание для сдававших зачёт 2019.02.23--2019.03.07 (8 баллов)

Докажите, что если $(a+b)$ делится $a$, то и $b$ делится на $a$.

\end{enumerate}

\end{document}
