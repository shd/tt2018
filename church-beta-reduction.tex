\documentclass[10pt,a4paper,oneside]{article}
\usepackage[utf8]{inputenc}
\usepackage[english,russian]{babel}
\usepackage{amsmath}
\usepackage{amsthm}
\usepackage{amssymb}
\usepackage{enumerate}
\usepackage{stmaryrd}
\usepackage[left=2cm,right=2cm,top=2cm,bottom=2cm,bindingoffset=0cm]{geometry}
\usepackage{proof}
\usepackage{url}

\begin{document}

\section*{О просто типизированном лямбда-исчислении по Чёрчу}

Типизированное по Карри исчисление использует для лямбда-выражений синтаксис, эквивалентный 
бестиповому лямбда-исчисленю, поэтому бета-редукция, теорема Чёрча-Россера и другие результаты
и конструкции можно без изменений взять из бестипового исчисления. Однако, исчисление по
Чёрчу имеет особый синтаксис для выражений, что требует от нас переформулировки определений
и передоказывания теорем.

\subsection{Синтаксис}

Фиксируем множество атомарных типовых переменных $V_T$ --- маленькие греческие буквы начала 
алфавита ($\alpha, \beta, \gamma, \dots$), возможно, с индексами и штрихами. 

Буквами из конца алфавита ($\sigma, \tau, \dots$) будем обозначать метапеременные для типов.
Будем говорить, что строчка является типом $\tau$, если выполнено одно из условий:

\begin{enumerate}
\item $\tau$ является атомарной переменной;
\item существуют два типа $\sigma$ и $\phi$, что $\tau = (\sigma\rightarrow\phi)$.
\end{enumerate}

Фиксируем множество атомарных переменных $V_\Lambda$ --- маленькие латинские буквы начала 
алфавита ($a, b, c, \dots$), возможно, с индексами и штрихами.
Буквами конца алфавита ($x, y, z$) будем обозначать метапеременные для атомарных переменных.

Большими буквами будем обозначать метапеременные для лямбда-выражений.
Будем говорить, что строчка $A$ является лямбда-предтермом по Чёрчу, если выполнено одно из условий:

\begin{enumerate}
\item $A$ является атомарной переменной из множества $V_\Lambda$;
\item существуют переменная $x$, тип $\tau$ и лямбда-выражение $P$, что $A = (\lambda x: \tau.P)$.
Указывать тип $x$ допустимо и с помощью верхнего индекса: $A = (\lambda x^\tau.P)$;
\item существуют выражения $P$ и $Q$, что $A = (P\ Q)$.
\end{enumerate}

Будем обозначать множество типов как $T$, и множество предтермов как $\underline{\Lambda_\texttt{ч}}$.

Несмотря на то, что мы потребовали обязательно указывать все скобки в выражениях, мы будем опускать
часть скобок в примерах, в предположении, что читатели без труда их расставят правильным образом.
Стрелка в типах правоассоциативна, тип $\alpha\rightarrow\beta\rightarrow\gamma$ следует
понимать как $(\alpha\rightarrow(\beta\rightarrow\gamma))$.
Правила расстановки скобок в лябмда-выражениях аналогичны таковым для бестипового исчисления: 
применение левоассоциативно; абстракция жадна и съедает столько, сколько может.

\subsection{Лямбда-термы}

Определим альфа-эквивалентность на лямбда-предтермах по Чёрчу.
Назовём терм $A$ альфа-эквивалентным терму $B$, если выполнено одно из следующих условий:

\begin{enumerate}
\item оба выражения --- одинаковые переменные: найдётся переменная $x$, что $A = x$ и $B = x$;
\item оба выражения --- применения эквивалентных выражений: найдутся $P_A$, $P_B$, $Q_A$ и $Q_B$, что
             $A = (P_A\ Q_A)$, $B = (P_B\ Q_B)$, $P_A =_\alpha P_B$, $Q_A =_\alpha Q_B$.
\item оба выражения --- абстракции эквивалентных выражений:
             найдутся $P_A$, $P_B$, $x$, $y$ и тип $\tau$, что $A = \lambda x: \tau.P_A$,
             $B = \lambda y: \tau.P_B$, и что если зафиксировать некоторую свежую переменную
             $t$, то $P_A[x := t] =_\alpha P_B[y := t]$.
\end{enumerate}

Построим фактор-множество по множеству $\underline{\Lambda_\texttt{ч}}$ с помощью отношения
альфа-эквивалентности, и назовём его множеством лямбда-термов по 
Чёрчу: $\Lambda_\texttt{ч} = \underline{\Lambda_\texttt{ч}}/(=_\alpha)$.

Определим множество свободных переменных лямбда-предтерма $A$ так:

$$FV(A) = \left\{ \begin{array}{ll} 
       \{x\}, & A = x \\
       FV(P)\cup FV(Q), & A = (P\ Q) \\
       FV(P)\setminus \{x\}, & A = (\lambda x:\sigma.P)            
\end{array}  \right.$$

Несложно показать, что данное множество не зависит от выбора представителя класса эквивалентности,
потому мы можем распространить данное определение и на лямбда-термы.

\subsection{Редукция}

Будем писать $A \rightarrow_\beta B$, если имеет место одна из следующих ситуаций:

\begin{enumerate}
\item $A = (\lambda x:\sigma.P)\ Q$, $B = P[x := Q]$;
\item $A = (P_A\ Q)$, $B = (P_B\ Q)$, причём $P_A\rightarrow_\beta P_B$;
\item $A = (P\ Q_A)$, $B = (P\ Q_B)$, причём $Q_A\rightarrow_\beta Q_B$;
\item $A = (\lambda x:\sigma.P_A)$, $B = (\lambda x:\sigma.P_B)$, причём $P_A\rightarrow_\beta P_B$.
\end{enumerate}

\noindent Обратите внимание, редукция не обращает внимание на типы. Если в процессе подстановки 
возникнет конфликт переменных (не будет выполнена свобода для подстановки), то требуется
произвести переименование связанных переменных и продолжить редукцию --- аналогично
бестиповому исчислению.

\subsection{Типизация}

\emph{Контекстом} выражения мы назовём список $\Gamma$, содержащий выражения вида $x : \sigma$.

\subsection{Вспомогательные леммы}

\subsection{Теорема Чёрча-Россера}

Пусть даны $A,B,C \in \Lambda_\texttt{ч}$, такие, что:
\begin{enumerate}
\item $\Gamma\vdash_\texttt{ч} A:\sigma$, $\Gamma\vdash_\texttt{ч} B:\sigma$, $\Gamma\vdash_\texttt{ч} C:\sigma$;
\item $B \ne_\alpha C$;
\item $A \rightarrow_\beta B$ и $A \rightarrow_\beta C$.
\end{enumerate}
Тогда существует такой $D$, что:
\begin{enumerate}
\item $\Gamma\vdash_\texttt{ч} D:\sigma$;
\item $B \rightarrow_\beta D$ и $C \rightarrow_\beta D$.
\end{enumerate}

\noindent Доказательство --- аналогично теореме Чёрча-Россера для бестипового исчисления.

Существует большое искушение доказывать теорему по следующему плану:
\begin{enumerate}
\item по теореме Чёрча-Россера для бестипового исчисления показать 
существование $D' \in \Lambda$, такого, что $|B| \rightarrow D'$ и $|C| \rightarrow D'$;
\item по леммам о редукции и поднятии заметить, что если $|B| \rightarrow_\beta D'$ и $\Gamma\vdash B:\sigma$, 
то существует $D$, такой, что $B \rightarrow_\beta D$, $|D| = D'$ и $\Gamma\vdash_\texttt{ч} D: \sigma$;
\item предположить, что этот $D$ и есть искомый.
\end{enumerate}

Однако, это рассуждение не гарантирует нам, что $C \rightarrow_\beta D$,
а только $|C| \rightarrow_\beta D'$, что не является доказательством требуемого утверждения.
Попытка сослаться ещё раз на лемму о редукции для $C$ даст нам существование какого-то $D_1$,
такого, что $C \rightarrow_\beta D_1$ и $|D_1|=D'$, но не докажет того, что $D = D_1$.

\end{document}
